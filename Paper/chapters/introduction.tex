\todo{Add a quote about Bayes' theorem in intro.}


“What information consumes is rather obvious: It consumes the
attention of its recipients. Hence a wealth of information creates a
poverty of attention, and a need to allocate that attention efficiently
among the overabundance of information sources that might
consume it.”
Herbert Simon (1971)


Try to fit Keynes' quote:
‘‘… professional investment may be likened to those newspaper competitions in which the competitors have to pick out the six prettiest faces from a hundred photographs, the prize being awarded to the competitor whose choice most nearly corresponds to the average preferences of the competitors as a whole; so that each competitor has to pick, not those faces which he himself finds prettiest, but those which he thinks likeliest to catch the fancy of the other competitors, all of whom are looking at the problem from the same point of view. It is not a case of choosing those which, to the best of one’s judgement, are really the prettiest, nor even those which average opinion genuinely thinks the prettiest. We have reached the third degree where we devote our intelligences to anticipating what average opinion expects the average opinion to be. And there are some, I believe, who practise the fourth, fifth and higher degrees.’’ Keynes (1936), page 156.


“It is not a case of choosing those [faces] which, to the best of one’s judgment, are really
the prettiest, nor even those which average opinion genuinely thinks the prettiest. We have
reached the third degree where we devote our intelligences to anticipating what average opinion expects the average opinion to be. And there are some, I believe, who practise the fourth,
fifth and higher degrees.” Keynes, General Theory of Employment Interest and Money, 1936


The beauty contest concept is issued not for asset pricing but for newspapers content (cf. Nimark)


As Shiller put it in his 2000 paper "Do ....", stocks volatility tends to be higher than what is predicted by the volatility of fundamentals. There is a broad literature (sources!) that tries to explain what is now called the "excess volatility puzzle". A related phenomenon attracted attention (at some point...), assets prices and their returns tend to comove more than expected by the comovement of their fundamentals.



\section{High level intro w/ most important/interesting findings/results}
\section{Description of paper content and summary of main results, by section}


Intro to the topic of comovement. Can use Shiller as example of price moving more than fundamentals and then link to similar observations in the comovement literature.
\vspace{1cm}

Introduction from "Institutional Clientele and Comovement - Zheng Sun"
One central question in asset pricing research is what drives stock comovement. Previous
literature has extensively documented that correlated cash flows and systematic shifts in
discount rates are important determinants of stock return comovement (eg. Fama and
French, 1993). But whether fundamentals are the only determinants is hotly debated.


\section{Literature Review}\todo{Go over short introduction of Deng2016}

A family of models which tries to explain the excess comovement of seemingly unrelated assets and their returns by grouping them based on non-fundamental similarities is that of the habitat framework. \textcite{Barberis2003} explain excess comovement of asset prices by the fact that some investors group risky assets in different styles and that they move funds across those style groups depending on their relative performance (this is akin to tactical allocation in factor investing). As funds move in and out of such style groups, assets will show excess comovement with those of the same group. Similarly, \textcite{Sun2008} focuses on institutional investors. But focuses on the institutional ownership among stocks. By a similar argument, institutional investors group together into distinct clienteles, based on similarities in their holdings. And the excess stock comovement is explained by the behavior of these groups. I.e. the focus is not put on strategies or factors but on groups (or clienteles) implicitly investing in the same set of factors or styles. \textcite{Barberis2005} find evidence that stocks begin to comove more excessively after being included in the S\&P 500, and vice-versa for stocks being excluded from the index. This family of models in which the excess comovement of seemingly unrelated assets and their returns is explained by a "grouping" is called "habitat framework".

\vspace{1cm}

\textcite{Barberis2005}, \textcite{Greenwood2008} --> stock returns exhibit strong comovement even when their fundamentals would not suggest so. 

\vspace{1cm}

A suggested explanation for excess comovement has been that it is due to institutional ownership. Ie. stocks with higher institutional ownership comove more (\cite{Pindyck1993}). In a more nuanced approach, \textcite{Kacperczyk2012} and \textcite{Kacperczyk2016} find evidence that only passive institutional ownership could explain the excess comovement of stock returns.

\vspace{1cm}

\textcite{Andersen2000} show that realized correlation of equity returns tend to be higher when realized volatility is high, and vice versa.

\vspace{1cm}

In the literature about assets pricing, portfolio formation and information choice \textcite{VanNieuwerburgh2010} do not allow agents to choose between idiosyncratic or aggregate information. \textcite{Mondria2010} uses only two risky assets, as a consequence all shocks are aggregate. \textcite{Kacperczyk2016} allow for N risky assets and a more general information structure, via a principal component decomposition of the signal structure.

This project pulls together material from three different but somewhat related strands of the economic literature.

\subsubsection{Measure of asymmetric correlation}
In the literature, asymmetric correlation refers to the fact that stock returns correlations tend to be higher in down market than in up\todo{up?} markets. We however focus not directly on thee direction of the market. We are more interested by the changes in correlations due to news.


In the literature, asymmetric correlation is often measured through models of the GARCH family (DCC, GARCH-M, regime-switching GARCH to mention a few popular ones). A few papers have tried different approaches.

Ang \& Chen (2002) develop a non-parametric measure of conditional correlation based on correlation behavior in the tails of the returns distribution. They define downside movements as those where both the market and an equity portfolio return are below pre-set levels. They find that downside correlation is higher than predicted by a normal distribution of returns and that upside correlation cannot be statistically distinguished from that predicted by a normal distribution. These results are similar to Longin \& Solnik (2001), who use extreme value theory to measure excess correlations in the tails of returns distribution of international equities. A term often used in this strand of literature is "exceedance correlation".

Forbes and Rigobon (2002) use a VAR model to measure cross-market correlations. They find that excess correlation is conditional on high volatility and that unconditional correlations are not higher in downturns than they are in normal times.

Gastaldi \& Nardecchia (2003) use a Kalman filter approach to estimate conditional time-varying beta of Italian industries.

Barberis, Shleifer \& Wurgler (2005) compare the beta of stock returns and the $R^2$ of a typical regression before and after a stock gets included in the S\&P 500. They find significant evidence of increased comovement after the inclusion and of significant decrease after the exclusion.

Deng (2016) proposes a modified state-space version of Kendall's tau and finds marginal evidence of asymmetric state-dependent comovement.
