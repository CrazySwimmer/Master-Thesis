\section{High level intro w/ most important/interesting findings/results}
\section{Description of paper content and summary of main results, by section}


Intro to the topic of comovement. Can use Shiller as example of price moving more than fundamentals and then link to similar observations in the comovement literature.
\vspace{1cm}

Introduction from "Institutional Clientele and Comovement - Zheng Sun"
One central question in asset pricing research is what drives stock comovement. Previous
literature has extensively documented that correlated cash flows and systematic shifts in
discount rates are important determinants of stock return comovement (eg. Fama and
French, 1993). But whether fundamentals are the only determinants is hotly debated.


\section{Literature Review}\todo{Go over short introduction of Deng2016}

A family of models which tries to explain the excess comovement of seemingly unrelated assets and their returns by grouping them based on non-fundamental similarities is that of the habitat framework. \textcite{Barberis2003} explain excess comovement of asset prices by the fact that some investors group risky assets in different styles and that they move funds across those style groups depending on their relative performance (this is akin to tactical allocation in factor investing). As funds move in and out of such style groups, assets will show excess comovement with those of the same group. Similarly, \textcite{Sun2008} focuses on institutional investors. But focuses on the institutional ownership among stocks. By a similar argument, institutional investors group together into distinct clienteles, based on similarities in their holdings. And the excess stock comovement is explained by the behavior of these groups. I.e. the focus is not put on strategies or factors but on groups (or clienteles) implicitly investing in the same set of factors or styles. \textcite{Barberis2005} find evidence that stocks begin to comove more excessively after being included in the S\&P 500, and vice-versa for stocks being excluded from the index. This family of models in which the excess comovement of seemingly unrelated assets and their returns is explained by a "grouping" is called "habitat framework".

\vspace{1cm}

\textcite{Barberis2005}, \textcite{Greenwood2008} --> stock returns exhibit strong comovement even when their fundamentals would not suggest so. 

\vspace{1cm}

A suggested explanation for excess comovement has been that it is due to institutional ownership. Ie. stocks with higher institutional ownership comove more (\cite{Pindyck1993}). In a more nuanced approach, \textcite{Kacperczyk2012} and \textcite{Kacperczyk2016} find evidence that only passive institutional ownership could explain the excess comovement of stock returns.

\vspace{1cm}

\textcite{Andersen2000} show that realized correlation of equity returns tend to be higher when realized volatility is high, and vice versa.

\vspace{1cm}

In the literature about assets pricing, portfolio formation and information choice \textcite{VanNieuwerburgh2010} do not allow agents to choose between idiosyncratic or aggregate information. \textcite{Mondria2010} uses only two risky assets, as a consequence all shocks are aggregate. \textcite{Kacperczyk2016} allow for N risky assets and a more general information structure, via a principal component decomposition of the signal structure.

This project pulls together material from three different but somewhat related strands of the economic literature. 