Usual conclusion

A next step would be to propose a formal framework to model the discussed behaviors. However, this is not without challenge. The typical agent in the current literature on rational inattention and costly information is allowed to choose higher the signal and/or the precision of such signal. Also, the distinction between private and public information is usually clear cut. Adding delegated information choice to the classical models of noisy rational expectation is challenging because (1) the distinction between public and private information is blurred (indeed, a published story reported by many information providers is considered private information if nobody learn about it, even though it is freely available), (2) delegated information choice adds a layer of filtering between the state of the world and the agents (ie. the editorial process) and as a consequence an agent might want to learn about something that has not been reported. \todo{footnote from \textcite{Nimark2019} could be useful here, to complement}